\documentclass[a4paper,12pt]{extarticle}

% Для поддержки русского языка
\usepackage[T2A]{fontenc}
\usepackage[utf8]{inputenc}
\usepackage[english, russian]{babel}

\usepackage{geometry} % Простой способ задавать поля
	\geometry{top=20mm}
	\geometry{bottom=20mm}
	\geometry{left=20mm}
	\geometry{right=20mm}

\usepackage{indentfirst} % Отступ первого абзаца

\usepackage{amsmath,amsfonts,amssymb,amsthm,mathtools} % Формулы
\usepackage{icomma} % "Умная" запятая: $0,2$ --- число, $0, 2$ --- перечисление

\usepackage{hyperref}
\hypersetup{
  unicode=true,
  colorlinks=true,
  linkcolor=blue,
  citecolor=teal,
  urlcolor=magenta
}
\usepackage{enumitem}
\usepackage{framed}
\usepackage{xcolor}
\definecolor{mygreen}{RGB}{0,160,0}

\author{Татаринова А.Г.}
\title{Лабораторная работа 1}
\date{\today}

\begin{document}

\maketitle

\text{Цель}: познакомиться с работой протокола HTTP, используя инструмент разработчика в браузере и утилиту curl. Получить навыки по созданию HTML страниц с CSS.
\begin{itemize}
    \item Лабораторная работа сдается преподавателю лично студентом.
    \item Сдача лабораторной работы заключается в том, что студент отвечает на вопросы преподавателя по заданиям, а также показывает и комментирует результаты самостоятельно выполненных заданий.
    \item Всё что написано в заданиях «изучите, узнайте и прочее» нужно изучить и для себя зафиксировать, так как преподаватель может спросить об этом при сдаче лабораторной работы.
\end{itemize}

\section{Протокол HTTP}
Изучите материал \url{https://practicum.yandex.ru/blog/chto-takoe-protokol-http/}.
\subsection{Задание}

Используя \textbf{руководство по вашему браузеру} и самостоятельную работу с браузером, изучите интерфейс и основные возможности встроенного в ваш браузер \textbf{инструмента разработчика} (обычно вызывается по нажатию клавиши F12). Например, для браузера Google Chrome можно начать изучение с чтения справки \url{https://developer.chrome.com/docs/devtools/overview}, особенно полезны панели "Элементы", "Сеть" и "Консоль".

Выполните на собственных примерах (можно использовать страницу некоторой статьи из Википедии):
\begin{enumerate}[label=\alph*)]
  \item Просмотр кода содержимого веб-страницы (HTML, CSS и JS).
  \item Поиск кода, который соответствует элементу на странице.
  \item Отследите HTTP-трафик страницы во время и после ее загрузки.
  \item Просмотр обнаруженных ошибок в коде страницы.
  \item Просмотр времени загрузки страницы.
  \item Просмотр версии протокола http, которая использовалась при получении веб\-ресурса.
\end{enumerate}


\subsection{Задание}

Изучите материал об утилите curl, используя, например, \url{https://cloud.vk.com/blog/10-komand-curl-kotorye-vam-sleduet-znat}.

\begin{enumerate}[label=\alph*)]
  \item Используя curl, получите заголовки HTTP-ответа при запросе страницы Википедии со статьей о городе Кирове.
  \item Используя curl, скачайте локально на компьютер файл pdf с расписанием вашей группы с сайта университета ВятГУ. По какому пути скачивается файл и как этот путь можно изменить?
\end{enumerate}

\subsection{Задание}
Познакомьтесь с библиотекой requests (язык python), которая позволяет выполнять HTTP-
запросы. Для этого прочитайте статью (\url{https://education.yandex.ru/handbook/python/article/modul-
requests}) из курса «Основы Python» (данный ресурс используйте для знакомства с языком Python).

Устанавливать библиотеку не требуется, \textbf{при необходимости обратитесь к преподавателю}.
\begin{enumerate}[label=\alph*)]
  \item Разберите и выполните пример из статьи по работе с API сервиса Яндекс Карты.
  \item Измените координаты в запросе на месте, где бы вы хотели побывать и получите новое изображение карты. Сохраните изображение.
  \item Выполните HTTP-запрос из предыдущего задания в браузере. Каков будет результат?
\end{enumerate}

\begin{leftbar}
  После выполнения запроса через requests возвращается объект, у которого есть атрибуты: \textbf{status\_code} - возвращает код выполнения HTTP-запроса (при успехе равен 200), \textbf{content} и \textbf{text} - отвечают за содержимое тела HTTP-ответа.
\end{leftbar}

\section{Язык HTML}
Изучите материалы:
\begin{enumerate}
  \item \url{https://practicum.yandex.ru/blog/zachem-nuzhen-html/}
  \item \url{https://practicum.yandex.ru/blog/chto-takoe-css/}
\end{enumerate}

\subsection{Задание}
Познакомьтесь с основами HTML и CSS, пройдя и выполнив задания \underline{без регистрации} первых \textbf{11 ГЛАВ} вводного курса \url{https://htmlacademy.ru/courses/297/run/1}.

Для отчета \textbf{сохраните} скриншоты выполняемых заданий. Главы устроены так, что сначала идёт теория, затем нужно "Перейти к заданию" и выполнить задания, которые располагаются в нижней правой части. Выполненные задания отмечаются \textcolor{mygreen}{зелёным цветом}.

\subsection{Задание}
Пройдите тест \url{https://use-web.ru/testpractice.php?action=html}.

Для отчета сохраните скриншот с результатом.

\end{document}