\documentclass[a4paper,12pt]{extarticle}

% Для поддержки русского языка
\usepackage[T2A]{fontenc}
\usepackage[utf8]{inputenc}
\usepackage[english, russian]{babel}

\usepackage{geometry} % Простой способ задавать поля
	\geometry{top=20mm}
	\geometry{bottom=20mm}
	\geometry{left=20mm}
	\geometry{right=20mm}

\usepackage{indentfirst} % Отступ первого абзаца

\usepackage{amsmath,amsfonts,amssymb,amsthm,mathtools} % Формулы
\usepackage{icomma} % "Умная" запятая: $0,2$ --- число, $0, 2$ --- перечисление

\usepackage{graphicx}  % Для вставки рисунков
\graphicspath{{images/}}

\usepackage{hyperref}
\hypersetup{
  unicode=true,
  colorlinks=true,
  linkcolor=blue,
  citecolor=teal,
  urlcolor=magenta
}
\usepackage{enumitem}
\usepackage{framed}
\usepackage{xcolor}
\definecolor{mygreen}{RGB}{0,160,0}

\author{Татаринова А.Г.}
\title{Лабораторная работа 3}
\date{\today}

\begin{document}

\maketitle

\text{Цель}: овладеть навыками семантической верстки с помощью средств css. Улучшить
умением использовать Flexbox и CSS Grid.
\begin{itemize}
    \item Лабораторная работа сдается преподавателю лично студентом.
    \item В ходе сдачи лабораторной работы показывается написанный код и даются по нему комментарии.
\end{itemize}


\section{Задание}
Сверстайте страницу «Афиша города»: шапка с навигацией, блок фильтров, сетка карточек событий, блок тегов, подвал. Используйте двухколоночную раскладку (фильтры слева, контент справа).
Требования и ограничения:
\begin{enumerate}
  \item Только HTML и CSS, без фреймворков и CSS-библиотек.
  \item Семантическая разметка: header, nav, main, aside, section, article, footer.
  \item Сетка страницы на CSS Grid, локальные компоненты и выравнивание —
на Flexbox.
  \item Сделайте переключатель темы (светлая/темная)
\end{enumerate}

Пример:
\begin{figure}[!ht]
    \centering
    \includegraphics[width=0.8\textwidth]{lab3_task1.png}
\end{figure}

\section{Задание}

Познакомьтесь с библиотекой Bootstrap (например, используя \href{https://itchief.ru/bootstrap}{ресурс}).
Официальный ресурс библиотеки Bootstrap - \href{https://getbootstrap.com}{ссылка}.

Добавьте в свой проект несколько компонентов из Bootstrap 5, не используя JavaScript. Добавьте или перепишите часть элементов интерфейса (например, карточки событий, кнопки, формы фильтрации, навигацию) с применением соответствующих Bootstrap-классов. \textbf{На странице должно быть не менее трёх компонентов библиотеки Bootstrap}.
Пример работы с компонентами:
\begin{enumerate}
  \item Карточки событий: Оформите каждую карточку с помощью класса .card, внутри используйте .card-body, .card-title, .card-text, и, при необходимости, .btn.
  \item Кнопки: Замените собственные классы кнопок на .btn и вариации .btn-primary, .btn-outline-secondary, и так далее.
  \item Формы фильтрации: Используйте классы .form-label, .form-select, .form-control вместо своих оформлений для <select>, <input>, <label>.
  \item Навигация: Перестройте меню с помощью .nav, .nav-pills или .nav-tabs — статические, без выпадающих списков и переключателей.
\end{enumerate}

\end{document}