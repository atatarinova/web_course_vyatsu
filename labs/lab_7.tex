\documentclass[a4paper,12pt]{extarticle}

% Для поддержки русского языка
\usepackage[T2A]{fontenc}
\usepackage[utf8]{inputenc}
\usepackage[english, russian]{babel}

\usepackage{geometry} % Простой способ задавать поля
	\geometry{top=20mm}
	\geometry{bottom=20mm}
	\geometry{left=20mm}
	\geometry{right=20mm}

\usepackage{indentfirst} % Отступ первого абзаца

\usepackage{amsmath,amsfonts,amssymb,amsthm,mathtools} % Формулы
\usepackage{icomma} % "Умная" запятая: $0,2$ --- число, $0, 2$ --- перечисление

\usepackage{graphicx}  % Для вставки рисунков
\graphicspath{{images/}}

\usepackage{hyperref}
\hypersetup{
  unicode=true,
  colorlinks=true,
  linkcolor=blue,
  citecolor=teal,
  urlcolor=magenta
}
\usepackage{enumitem}
\usepackage{tcolorbox}
\usepackage{framed}
\usepackage{xcolor}
\definecolor{mygreen}{RGB}{0,160,0}

\author{Татаринова А.Г.}
\title{Лабораторная работа 7 \\
    \large Дополнительная}
\date{\today}

\begin{document}

\maketitle

\textbf{Цель}: получить навыки написания клиентской стороны с запросом к API. 
\vspace{\baselineskip}
\begin{tcolorbox}
\textbf{Важное замечание}: Лабораторная работа является \underline{дополнительной}. Её выполнение засчитывается за сдачу финального теста на последнем лабораторном занятии.
\end{tcolorbox}

\begin{itemize}
    \item Лабораторная работа сдается преподавателю лично студентом.
    \item Сдача лабораторной работы заключается в том, что студент отвечает на вопросы преподавателя по заданиям, а также показывает и комментирует результаты самостоятельно выполненных заданий.
    \item Всё что написано в заданиях «изучите, узнайте и прочее» нужно изучить и для себя зафиксировать, так как преподаватель может спросить об этом при сдаче лабораторной работы.
\end{itemize}

В качестве вспомогательного материала можно использовать:
\begin{itemize}
    \item Уроки курса \href{https://metanit.com/web/vue/1.1.php}{«Основы Vue 3»} с ресурса Metanit.
    \item \href{https://ru.vuejs.org/guide/introduction.html}{Официальное руководство} (для удобства можно установите язык "Русский" в главном меню на странице).
\end{itemize}


\section{Задание}
Напишите html страницу, на которой происходит взаимодействие с сервисом поиска статей, написанном в прошлой лабораторной работе, через функцию fetch() на ванильном javascript.

Взаимодействие с сервисом ограничивается реализацией поиска статей по их заголовкам и отображением результатов на этой же странице с использованием оформления CSS стиля.

\section{Задание}
Напишите веб-страницу, на которой через Vue.js происходит взаимодействие с сервисом поиска статей из прошлой лабораторной работы.

Взаимодействие с сервисом ограничивается реализацией поиска статей по их заголовкам и отображением результатов на этой же странице с использованием оформления CSS стиля.

\end{document}