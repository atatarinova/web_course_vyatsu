\documentclass[a4paper,12pt]{extarticle}

% Для поддержки русского языка
\usepackage[T2A]{fontenc}
\usepackage[utf8]{inputenc}
\usepackage[english, russian]{babel}

\usepackage{geometry} % Простой способ задавать поля
	\geometry{top=20mm}
	\geometry{bottom=20mm}
	\geometry{left=20mm}
	\geometry{right=20mm}

\usepackage{indentfirst} % Отступ первого абзаца

\usepackage{amsmath,amsfonts,amssymb,amsthm,mathtools} % Формулы
\usepackage{icomma} % "Умная" запятая: $0,2$ --- число, $0, 2$ --- перечисление

\usepackage{graphicx}  % Для вставки рисунков
\graphicspath{{images/}}

\usepackage{hyperref}
\hypersetup{
  unicode=true,
  colorlinks=true,
  linkcolor=blue,
  citecolor=teal,
  urlcolor=magenta
}
\usepackage{enumitem}
\usepackage{tcolorbox}
\usepackage{framed}
\usepackage{xcolor}
\definecolor{mygreen}{RGB}{0,160,0}

\author{Татаринова А.Г.}
\title{Лабораторная работа 6}
\date{\today}

\begin{document}

\maketitle

\text{Цель}: получить навыки написания веб-сервиса на Fast-API.
\begin{itemize}
    \item Лабораторная работа сдается преподавателю лично студентом.
    \item Сдача лабораторной работы заключается в том, что студент отвечает на вопросы преподавателя по заданиям, а также показывает и комментирует результаты самостоятельно выполненных заданий.
    \item Всё что написано в заданиях «изучите, узнайте и прочее» нужно изучить и для себя
зафиксировать, так как преподаватель может спросить об этом при сдаче
лабораторной работы.
\end{itemize}
\textbf{(!)} В качестве дополнительного руководства можно использовать уроки курса \href{https://metanit.com/python/fastapi/1.12.php}{«Python и FastAPI»} с ресурса Metanit.


\section{Задание}
Изучите материалы 1, 2, 3 и 4 глав книги «Создание веб-API Python с помощью
FastAPI». Реализуйте пример приложения ToDo по материалам книги только этих глав (Часть 1).

\section{Задание}
Запустите скрипт app.py в каталоге server проекта \textbf{web-lab-6-proj-test}. Через UI Swagger вызовите документацию для api запущенного сервиса через \href{localhost:8085}{localhost:8085} (надо добавить к адресу «docs»). Попробуйте вызвать каждый из маршрутов сервиса.

Также откройте веб-страницу index.html в каталоге client и проверьте работу соответствующего маршрута.

\section{Задание}
Напишите веб-сервис, выполняющий поиск научных статей ресурса arxiv.org по их заголовкам.

\begin{tcolorbox}
Для извлечения данных со страницы используйте библиотеку BeautifulSoup4 (bs4). Ниже показана краткая справка по извлечению данных со страницы.
\end{tcolorbox}

\begin{tcolorbox}
Для получения списка статей из области компьютерных наук выполняйте запросы, используя библиотеку requests (см. первую лабораторную работу). В своих запросах используйте параметр, хранящий запросные слова. Чтобы понять какой параметр использовать, обратите внимание на параметры адресной строки при поиске на странице \href{https://arxiv.org/search/cs}{https://arxiv.org/search/cs}.
\end{tcolorbox}

Приложение должно выполнять следующие функции:
\begin{enumerate}[label=\alph*)]
\item получение списка статей из области компьютерных наук (\href{https://arxiv.org/search/cs}{https://arxiv.org/search/cs}), в заголовках (title) которых встречаются заданные пользователем слова (параметры в маршруте),
\item получение названия, авторов и аннотации статьи по её ID на сайте arxiv.org (уточнение смотри ниже на поясняющих скриншотах),
\item перевод аннотации выбранной статьи. Пример перевода текста можно посмотреть в коде для второго задания, в котором используется библиотека translate на основе Microsoft Translator (устанавливается как \textbf{pip install translate}). Учтите ограничение на длину переводного текста, для этого можно текст аннотации делить на несколько предложений и для них вызывать функцию перевода.
\end{enumerate}

Протестируйте работу веб-сервиса через UI Swagger.

\begin{tcolorbox}
URL страницы с аннотацией статьи составляется по правилу:\\
arxiv.org/abs/{ID статьи}.\\
Например: \href{https://arxiv.org/abs/2502.10299}{https://arxiv.org/abs/2502.10299}
\end{tcolorbox}

ID номером статьи является набор цифр, следующий после \textbf{arXiv:} в строке перед названием научной статьи:
\begin{figure}[!ht]
    \centering
    \includegraphics[width=0.8\textwidth]{lab6_task3.png}
\end{figure}

Краткая справка по использованию библиотеки BeautifulSoup4 (bs4):
\begin{figure}[!ht]
    \centering
    \includegraphics[width=0.8\textwidth]{lab6_task3_2.png}
\end{figure}
\begin{figure}[!ht]
    \centering
    \includegraphics[width=0.8\textwidth]{lab6_task3_3.png}
\end{figure}
\begin{figure}[!ht]
    \centering
    \includegraphics[width=0.8\textwidth]{lab6_task3_4.png}
\end{figure}
\begin{figure}[!ht]
    \centering
    \includegraphics[width=0.8\textwidth]{lab6_task3_5.png}
\end{figure}
\begin{figure}[!ht]
    \centering
    \includegraphics[width=0.8\textwidth]{lab6_task3_6.png}
\end{figure}

\end{document}